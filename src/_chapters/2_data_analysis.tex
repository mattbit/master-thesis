% !TEX root = ../thesis.tex

\chapter{Data analysis}\label{sec:data_analysis}


\section{A statistical framework for SPT data}

Modern imaging methods such as \tsc{palm} or \tsc{storm} have made possible to track simultaneously multiple particles at high spatiotemporal resolution. Single-particle tracking (\tsc{spt}) techniques can then be used to reconstruct thousands of particle trajectories. To extract meaningful insights from this huge amount of data, we need a modelling and statistical framework.

The data presented here was provided by the Molecular Neuroscience Group at the University of Cambridge, \tsc{uk}. It consists into trajectory fragments of amyloid beta aggregates observed in \tsc{HEK 293T} cells. Nevertheless, the statistical methods described in the following provide a general framework to extract dynamical parameters of particles at cell scale. The \tsc{spt} trajectories were provided by the Molecular Neuroscience Group and double checked by also reconstrucing them from imaging data using the TrackMate software \cite{trackmate}. The tools developed to extract biophysical features were implemented in a Python package \cite{sptx}, which allows to flexibly analyse new datasets.
%
%
% \subsection{Effective dynamical model}
%
% At the microscopic level, the motion of the molecules can be described by Langevin's equation. In the case of biological processes we are interested in, the Langevin dynamics can be considered in its large friction limit (Smoluchowski's equation) \cite{hoze2012, hoze2014}
% \begin{equation} \label{eq:smoluchowski}
%  \dot{\bm{x}} = \frac{\bm{F}(\bm{x})}{\gamma} + \sqrt{2D} \dot{\bm{w}}
% \end{equation}
% where $\bm{F}(\bm{x})$ is the drift force exerted on the particle at position $\bm{x}$, $\gamma$ is the friction coefficient, $D$ is the diffusion coefficient and $\bm{w}(t)$ is a two-dimensional Wiener process. At this scale, it makes sense to consider the diffusion to be mainly due to thermal agitation so that it can be considered isotropic.
%
% However, it is not possible to directly recover the microscopic model from the \tsc{spt} data, since we miss information about the local behaviour both in space (such as the presence of microscopic obstacles undetected by the imaging device) and time (such as thermal fluctuations much faster than the acquisition timescale). We can still build a coarse-grained model \cite{hoze2012, hoze2014}, transforming eq. \ref{eq:smoluchowski} into the effective stochastic equation
% \begin{equation} \label{eq:effective}
%  \dot{\bm{x}} = \bm{a}(\bm{x}) + \sqrt{2}\bm{B} \dot{\bm{w}}
% \end{equation}
% where $\bm{a}(\bm{x})$ is the effective velocity field and $\bm{D} \equiv \bm{B}^T\bm{B}$ is the effective diffusion tensor. It must be noted that, in principle, the effective diffusion coefficient may not be isotropic since it takes into account the local microscopic features (e.g. obstacles). On the other hand, actual analysis of the data show that anisotropic components are negligible: in the following, the diffusion tensor is reduced for simplicity to a scalar coefficient by averaging on the diagonal entries. Moreover, the velocity field is assumed time invariant in the relatively short time window spanned by the \tsc{spt} data (30--60 s).
%
% \begin{figure}
% 	\begin{subfigure}[l]{5.33cm}
% 		\caption{}\label{fig:velocity_smoothening_original}
% 		\includegraphics{spt/velocity_field_original.pdf}
% 	\end{subfigure}
% 	\begin{subfigure}[l]{5.33cm}
% 		\caption{}\label{fig:velocity_smoothening_smooth}
% 		\includegraphics{spt/velocity_field_smooth.pdf}
% 	\end{subfigure}
% 	\begin{subfigure}[l]{5.1cm}
% 		\caption{}\label{fig:cdr}
% 		\includegraphics{spt/velocity_field_regions.pdf}
% 	\end{subfigure}
% 	\begin{subfigure}[l]{0cm}
% 	    \vspace{\baselineskip}
% 		\includegraphics{spt/velocity_field_colorbar.pdf}
% 	\end{subfigure}
%    	\captionsetup{width=\widewidth}
% 	\caption{
% 		Velocity field smoothening on a grid of size \SI{250}{nm}. \textbf{\subref{fig:velocity_smoothening_original}} original velocity field, \textbf{\subref{fig:velocity_smoothening_smooth}} velocity field after the smoothening,
% 		\textbf{\subref{fig:cdr}} Regions of coherent motion on a grid of size \SI{500}{nm}. The angular distance threshold was set to \SI{0.08}{} and clusters smaller than 10 bins were discarded. The area color indicates the mean angle of the velocity among the cluster.
% 	}
% 	\label{fig:velocity_smoothening}
% \end{figure}
%
%
% \subsection{Diffusion and velocity field estimation}
%
% To estimate the dynamical parameters of eq. \ref{eq:effective} a statistical analysis is needed. We follow the approach demontrated in \cite{hoze2012, hoze2014, hoze2017, holcman2015} by partitioning the data in a square grid with fixed bin size. The velocity field and diffusion coefficients are considered constant in each bin.
% If the acquisition time interval is sufficiently small, the process described by eq. \ref{eq:effective} can be discretized following a forward Euler scheme:
% \begin{equation}
%  \bm{x}_{t+1} - \bm{x}_{t} = \bm{a}_{\Delta t}(\bm{x}_t) \Delta t + \sqrt{2\Delta t\bm{D}_{\Delta t}(\bm{x}_t)}  \bm{\eta}_t
% \end{equation}
% where $\Delta t$ is the acquisition time interval.
%
% Then, the trajectory step $\Delta\bm{x}_t \equiv \bm{x}_{t+1} - \bm{x}_t$ starting in bin $B$ is a normally distributed random variable with mean $\bm{a}(\bm{x}_B)$ and variance $2D(\bm{x}_B)\Delta t$ where $\bm{x}_B$ is the center of the bin. The velocity field and diffusion tensor can thus be recovered by computing the empirical estimate of the moments:
% \begin{align}
%  \bm{a}_{\Delta t}(\bm{x}_B) & = \frac{1}{\Delta t} \mathbb{E}_B\left[\Delta \bm{x}\right]                                                   \\
%  \bm{D}_{\Delta t}(\bm{x}_B) & = \frac{1}{2\Delta t} \left( \mathbb{E}_B\left[\Delta \bm{x}^2\right] - \mathbb{E}_B[\Delta \bm{x}]^2 \right)
% \end{align}
% where the expected value is taken on the bin. Notice that the $\bm{a}_{\Delta t}$ and $\bm{D}_{\Delta t}$ are not the same of those of eq. \ref{eq:effective}, as they depend on the acquisition interval $\Delta t$. In fact, we may recover the coefficients of the continuous process only in the limit $\Delta t \to 0$. To avoid a too cluttered notation, we will drop the $\Delta t$ index in the following and assume it implicitly.
%
%
% \subsection{Field smoothening}\label{sec:smoothening}
%
% To perform analysis and simulations we often need a smooth representation of the velocity or diffusion field, providing estimates also for bins with few—or missing—data. To solve this problem, a convolution with variable kernel is applied on the field. Denoting by $n_B$ the number of datapoints inside bin $B$, the smoothened field $\tilde{\bm{a}}$ is obtained as
% \begin{equation} \label{eq:convolution}
%  \bm{\tilde{a}}(\bm{x}_B) = (k \ast a)(\bm{x}_B) \equiv \frac{\displaystyle \sum_{S \in \Gamma(B)} n_S \bm{a}(\bm{x}_S)}{\displaystyle \sum_{S' \in \Gamma(B)} n_{S'}}
% \end{equation}
% where $\Gamma(B)$ is the Moore neighbourhood of $B$.
%
% Explicitly, representing bins with their grid site indices, the kernel matrix $K(i, j)$ centered in the bin $i, j$ is:
% \begin{equation}
%  K(i, j) = \frac{1}{\displaystyle \sum_{k = i-1}^{i+1}\sum_{l = j-1}^{j+1} n_{k,l}}
%  \begin{bmatrix}
%   n_{i-1,j-1} & n_{i-1, j} & n_{i-1, j+1} \\
%   n_{i,j-1}   & n_{i, j}   & n_{i, j+1}   \\
%   n_{i+1,j-1} & n_{i+1, j} & n_{i+1, j+1}
%  \end{bmatrix}
% \end{equation}
% Finally, bins such that the total number of samples in their Moore neighbourhood is lower than a threshold are excluded from the analysis. A comparison between the velocity field before and after the smoothening is visible in \cref{fig:velocity_smoothening_original,fig:velocity_smoothening_smooth}.
%
% This approach can be interpreted in terms of Bayesian inference, where the prior (Gaussian) distribution of the parameter of interest is based on the distributions in the neighbouring bins. The mean estimator on the posterior is then obtained by averaging the means of the neighbours with a weight proportional to the number of samples.
%
% \subsection{Regions of coherent motion}
% \label{sec:regions_coherent_motion}
%
% Regions of coherent motion are identified by clustering adjacent bins based on the angular distance of the velocity vectors. The angular distance between two vectors $\bm{u}$ and $\bm{v}$ is defined as:
% \begin{equation} \label{eq:angular-similarity}
%  \mathrm{distance}(\bm{u}, \bm{v}) \equiv \frac{1}{\pi} \arccos\left(\frac{\bm{u} \cdot \bm{v}}{\|\bm{u}\| \|\bm{v}\|}\right)
% \end{equation}
%
% \noindent A very simple algorithm works as follows:
% \begin{enumerate}
%  \begin{item}
%        Create an empty cluster (i.e. a set of adjacent bins) for each bin in the grid and assign the bin to it.
% \end{item}
% \begin{item}
%       For every bin $B$, consider each neighbouring bin $S \in \Gamma(B)$. If the angular distance between $\tilde{\bm{a}}(\bm{x}_B)$ and $\tilde{\bm{a}}(\bm{x}_S)$ is lower than a given threshold, merge the clusters of $B$ and $S$.
% \end{item}
% \end{enumerate}
%
% \noindent This extremely simple clustering method is quite effective in localising correlated motion and can be easily extended to also consider time evolution. An example of the result can be seen in figure \ref{fig:cdr}. The identification of regions of coherent motion can have relevant application in many biological contexts. As an example, coherent motion regions have been observed in interphase chromatin \cite{zidovska2013}, yielding new hypothesis about its biological functions.
%
%
% \subsection{Potential wells}
%
% To identify local attractors we use the methodology and formalism developed in \cite{hoze2012, parutto_wells}. Considering the velocity field to be locally conservative, it can be described by the gradient of a scalar potential:
% \begin{equation}
%  \bm{a}(\bm{x}) = - \nabla U(\bm{x})
% \end{equation}
% Local attractors can then be seen as wells in the potential. At first non-zero order around a local minumum $\bm{x}_0$ the potential is a paraboloid:
% \begin{equation} \label{eq:paraboloid}
%  U(x, y) = U_0 + A\left[\frac{(x - x_0)^2}{r_x^2} + \frac{(y - y_0)^2}{r_y^2}\right] + O(x, y)^2
% \end{equation}
% in a coordinate system $(x, y)$ where the axes are rotated by an angle $\varphi$ to match those of the paraboloid. A potential well is thus by the set of parameters $A$ (well depth), $\bm{x}_0$ (centre), $r_1$, $r_2$ (axes of the ellipse obtained by cutting the potential well at height $A$), and $\varphi$ (the angle of the ellipse major axis).
%
% If the diffusion coefficient is assumed to be locally constant, the particle density is described by the Boltzmann distribution
% \begin{equation}
%  \rho(x, y) \propto \exp\left(-\frac{U(x, y)}{D}\right)
% \end{equation}
% Substituting eq. \ref{eq:paraboloid}, the density around a local minimum is approximately Gaussian. Thus, after selecting a small high density region, principal component analysis (\tsc{pca}) can be used to approximate the location of the attractor ($\bm{x}_0$) and the ellipse parameters ($\varphi$, $r_1$, $r_2$) corresponding to the 95\% confidence ellipse.
%
% To find the remaining parameter $A$ an iterative procedure is used. A grid centered in $\bm{x}_0$ is built, and for each iteration $k$ an ellipse $\mathcal{E}_k$ with increasingly longer axis is obtained by rescaling the original confidence ellipse. At each iteration, we calculate the $A_k$ that minimize the mean squared error (\tsc{mse}) with respect to the velocity field of all the bins inside $\mathcal{E}$:
% \begin{equation}
%  \mathrm{MSE}_k = \sum_{\bm{x}_i \in \mathcal{E}_k} \| -\nabla U(\bm{x}_i) - \bm{a}(\bm{x}_i) \|^2
% \end{equation}
% The best fit for $A$ is then the $A_k$ corresponding to the iteration with minimal \tsc{mse}.
% A parabolic error score $S$ indicating how much the potential well resembles a paraboloid is defined as
% \begin{equation}
%  S \equiv \frac{\mathrm{MSE}_k}{\sum_{\bm{x}_i \in \mathcal{E}_k} \|\bm{a}(\bm{x}_i)\|^2}
% \end{equation}
% where $S \in [0, 1]$ and $S = 0$ indicates a perfect fit.
% The initial high density regions are localized by using the \tsc{dbscan} clustering algorithm.
%
%
% \subsection{Linking coherent motion regions to potential wells}
%
% It is possible to obtain a rough dynamical map of the cell by connecting the regions of coherent motion and the potential wells to form a directed graph.
% First, the axis corresponding to the average direction of each region is computed; then the endpoints of the regions are connected to other endpoints or potential wells that are found in a small neighbourhood. The result is visible in figure \ref{fig:graph}.
%
% This can provide a new way to understand the macroscopic dynamical structure of the cell, relative to the observed molecule.
%
% \begin{figure}
% 	\begin{wide}
% 	\begin{minipage}[t]{10.67cm}
% 	    \begin{subfigure}[t]{5.33cm}
% 	      \caption{}\label{fig:graph_regions}
% 	      \includegraphics{img/spt/fig/06_graph_regions.png}
% 	    \end{subfigure}
% 	    \begin{subfigure}[t]{5.33cm}
% 	      \caption{}\label{fig:graph_skeleton}
% 	      \includegraphics{img/spt/fig/06_graph_sk.png}
% 	    \end{subfigure}
% 	\end{minipage}
% 	\begin{minipage}[t]{5.3cm}
% 	    \vspace{\baselineskip}
% 		\caption{
% 		    Reconstruction of dynamical structure.
% 		    \textbf{\subref{fig:graph_regions}} coherent motion regions with main axis highlighted;
% 		    \textbf{\subref{fig:graph_skeleton}} graph reconstructed by linking potential wells and regions of coherent motion.
% 		  }
% 		  \label{fig:graph}
% 	\end{minipage}
% \end{wide}
% \end{figure}
%
% \subsection{Multiscale analysis}
%
% To investigate how the scale influences the detection of the features, the methods described in the previous sections were applied on different grid sizes. Two different datasets were analysed using bin sizes between \SI{100}{\nano\meter} and \SI{500}{\nano\meter}. Then, characteristics of both regions of coherent motion and potential wells were compared at different scales. An overview is shown in figure \ref{fig:multiscale_overview}.
%
% \begin{figure}[p]
% \begin{wide}
%     \begin{subfigure}[l]{5.33cm}
%     	\caption{\SI{100}{\nano\meter} grid}\label{fig:ms_wells_100}
%     	\includegraphics{img/spt/multiscale_100nm_wells.pdf}
%     \end{subfigure}
%     \begin{subfigure}[l]{5.33cm}
%       	\caption{\SI{300}{\nano\meter} grid}\label{fig:ms_wells_300}
%      	\includegraphics{spt/multiscale_300nm_wells.pdf}
%     \end{subfigure}
%     \begin{subfigure}[l]{5.33cm}
%     	\caption{\SI{500}{\nano\meter} grid}\label{fig:ms_wells_500}
%     	\includegraphics{spt/multiscale_500nm_wells.pdf}
%     \end{subfigure}
%
%     \par\medskip
%
% 	\begin{subfigure}[l]{5.33cm}
% 		\caption{}\label{fig:ms_cdr_100}
%         \includegraphics{spt/multiscale_100nm_regions.pdf}
%     \end{subfigure}
%     \begin{subfigure}[l]{5.33cm}
%       \caption{}\label{fig:ms_cdr_300}
%       \includegraphics{spt/multiscale_300nm_regions.pdf}
%     \end{subfigure}
%     \begin{subfigure}[l]{5.33cm}
%       \caption{}\label{fig:ms_cdr_500}
%       \includegraphics{spt/multiscale_500nm_regions.pdf}
%     \end{subfigure}
%
%     \par\medskip
%
%     \begin{subfigure}[l]{8cm}
%       \caption{}\label{fig:ms_cdr_count}
%       \includegraphics{spt/multiscale_regions_count.pdf}
%     \end{subfigure}
%     \begin{subfigure}[r]{8cm}
%       	\caption{}\label{fig:ms_cdr_area}
%       	\includegraphics{spt/multiscale_regions_area.pdf}
%     \end{subfigure}
%
%     \par\medskip
%
% 	\begin{subfigure}[l]{8cm}
% 		\caption{}\label{fig:ms_wells_count}
% 		\includegraphics{spt/multiscale_wells_count.pdf}
% 	\end{subfigure}
% 	\begin{subfigure}[r]{8cm}
% 	  	\caption{}\label{fig:ms_wells_score}
% 	  	\includegraphics{spt/multiscale_wells_score.pdf}
% 	\end{subfigure}
%
% 	\captionsetup{width=\widewidth}
% 	\caption{
% 		Multiscale overview on two datasets (D1, D2). \textbf{\subref{fig:ms_wells_100}, \subref{fig:ms_wells_300}, \subref{fig:ms_wells_500}} potential well detections in a subregion of D2;
% 		\textbf{\subref{fig:ms_cdr_100}, \subref{fig:ms_cdr_300}, \subref{fig:ms_cdr_500}} regions of coherent motion in a subregion of D2;
% 		\textbf{\subref{fig:ms_cdr_count}} number of regions of coherent motion detected;
% 		\textbf{\subref{fig:ms_cdr_area}} distribution of the areas of the regions of coherent motion;
% 		\textbf{\subref{fig:ms_wells_count}} number of potential wells with error score lower than 0.75;
% 		\textbf{\subref{fig:ms_wells_score}} average error score of potential wells at different grid sizes.
% 	}\label{fig:multiscale_overview}
% \end{wide}
% \end{figure}
%
%
% \subsection{Discussion}
%
% The analysis has shown that there is no well defined pathway along which the Aβ aggregates are driven. Instead, the molecules follow complex itineraries, bouncing between different regions. The reconstruction of the network of attractors and regions of coherent motion of figure \ref{fig:graph} gives an idea of this underlying structure. It is thus not possible to identify a clear directionality or an accumulation region in the global motion. This may suggest that the processing of Aβ is delocalized and molecules are bounced between several loosely distributed processing units.
%
% Multiscale analysis reveals that regions of coherent velocity, given their non-locality, tends to persist across different choices of the grid size. Yet, both datasets show a plateau in the number of coherent velocity regions (\cref{fig:ms_cdr_count}) for grid sizes between \SI{300}{\nano\meter} and \SI{400}{\nano\meter}. This gives a hint about their actual scale. As the grid size is increased the clusters expand (\cref{fig:ms_cdr_area}, but they retain the network structure of figure \ref{fig:graph} as no prevalent directionality emerges (\cref{fig:ms_cdr_100,fig:ms_cdr_300,fig:ms_cdr_500}).  The localization of potential wells is instead very sensible to the grid size. This is expected and confirms their local nature. As the grid size is reduced, more and more local minima of the potential are found (\cref{fig:ms_wells_count}). The average error score of the wells is slightly decreasing as the grid size increases (\cref{fig:ms_wells_score}). This probably due the higher noise present at small grid sizes, where the number of datapoints in each bin is smaller.

\section{Attractors}
\section{Pathways}
