% !TEX root = ../thesis.tex

\chapter{Conclusions}


\section{Perspective and open problems}

By means of statistical methods and modelling, in section \ref{sec:data_analysis}, we have tackled the Alzheimer's disease from the molecular point of view, building a structured representation of the amyloid-β dynamics starting from single-particle tracking data.
Unfortunately, the structure that this analysis has revealed is not of immediate nor simple interpretation.
Regions of coherent motion and attractors form a complex framework, suggesting that amyloid-β aggregates are processed by a delocalised system.
New results for the dynamics of the endoplasmic reticulum\mcite{holcman2018single}, also based on \tsc{spt} analysis, provided hints that this delocalised system may be represented by the \tsc{ER}.
These findings were exploited to build a model of motion on the \tsc{er} and, thanks to that, we have been able to provide an estimate for the timescale of the \tsc{ER} transport mechanism through numerical simulations.
However, values for the mean first passage time between nodes of the network found in the simulations are unexpectedly high (in the order of tens of minutes), hinting at the fact that \tsc{MFPT} may not be the right parameter to consider when evaluating the redistribution timescale. Efficient transport in the context of biological processes should probably exploit other means.

Speculating, we propose that one should look at extreme statistics to highlight the characteristics of this transport mechanism.
Considering an activation process where multiple particles are released from a source but few (or even just one) are sufficient to enable receptors, we focused on the time required for the first particle to arrive in a target node and found a much shorter timescale (in the order of seconds), that seems compatible with the class of biological processes we want to describe.
Hence, we compared our results with previous works dealing with an equivalent Brownian process\mcite{basnayake2017,basnayake2018,holcman2018asymptotics}, finding that the asymptotics for the first arrival time obtained in continuous space also hold in the case of motion on a network, and providing new evidence supporting the idea that fastest trajectories follow the spatially optimal path.
Our simulations of the active network model also revealed a novel transport mechanism which causes the particles to be delivered in redundant packets, supporting the idea that redundancy naturally emerges as a way to guarantee efficiency and robustness of biological processes, as it has already been pointed out in previous works\mcite{reynaud2015,schuss2017}.

Overall, we have explored a vast and hitherto little known domain, combining data analysis, modelling and numerical simulations, but most of our results remain open hypotheses: more experimental data (and more analysis) is needed to verify if the validity of the \tsc{ER} model we proposed and to highlight possible correlation between amyloid-β and \tsc{ER} dynamics; yet we offered a new direction and a new point of view to tackle these questions. Lastly, we hope that a better understanding of the \tsc{ER} transport mechanism will shed new light on amyloid-β aggregation and its clinical implications.
