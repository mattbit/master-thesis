% !TEX root = ../thesis.tex

\chapter{Introduction}\label{sec:introduction}

Super-resolution microscopy has become, with recent developments, a fundamental source of data in the field of biology. On top of this, imaging techniques like \textsc{palm} have been combined with single-particle tracking \cite{sptpalm}, allowing to reconstruct thousands of simultaneous particle trajectories at increasingly high spatiotemporal resolution. With the availability of this large quantity of information, new data analysis techniques are needed to extract relevant biophysical features that can give new insights on the biological processes. Existing knowledge from physics and mathematics can provide a solid base to build a toolset of statistical and analytical methods to accomplish this goal. Langevin's equation and stochastic processes have been extensively---and successfully---used to model the dynamics of biological systems \cite{schuss2010}. Likewise, regression and unsupervised learning algorithms represent a powerful tool to classify the large amount of data produced by single-particle tracking. Valuable insights can be obtained by carefully joining data driven methods and theoretical modelling. Either approaches are valid but may not be sufficient, per se, to reveal the inner significance of certain processes; their combination can instead provide hints about the deeper mechanisms of life. Redundancy, for example, seems to be a fundamental principle in many biological processes, as the joint application of narrow escape model and numerical simulations has suggested \cite{reynaud2015, schuss2017, basnayake2018}. Yet to be understood problems in biology can benefit hugely from this twofold approach. Gathering of high resolution datasets paves the way to new and promising discoveries. On the other hand, data is useless without a solid framework to analyse and explain it. Overwhelming availability of data can even be counterproductive, as it makes the extraction of meaningful patterns more complicated. Dealing with these challenges is going to be, in the very near future, one of the major efforts in all fields of scientific research.

In this thesis I will present some results regarding data analysis, modelling and numerical simulations. The \textit{fil rouge} connecting the whole work is the search for explanation of the dynamics of amyloid beta (Aβ) aggregates in neurons, which are known to be involved in Alzheimer's disease \cite{AD}. In this introduction I will briefly review the current knowledge about Alzheimer disease and single-particle tracking methods. In chapter \ref{sec:data_analysis} I will present the analysis of \textsc{spt} data. While the dataset used specifically regards Aβ aggregates, the final aim is to build a general methodological framework to allow the extraction of biophysical features from \textsc{spt} recordings. In chapter \ref{sec:modelling} I will instead focus on a model of motion involving the endoplasmic reticulum (\textsc{er}) that has been developed by Holcman's team \cite{parutto_er} based on the analysis of the Aβ data, describing the results of numerical simulations that I developed to better understand the biological implications of the model.

\section{Alzheimer's disease}

Alzheimer’s disease is the most common form of dementia in elder adults. As population ages, the death rate due to the disease is steadly increasing. Statistics published by the Centres for Disease Control and Prevention show that Alzheimer’s-related deaths in \tsc{USA} increased 55\% between 1999 and 2014, making it the sixth leading cause of death. The disease causes a progressive loss of brain functions (neuron deaths) to a point where it impaires basic cognitive skills. Most common symptoms are memory loss and language impairement.

As of today, causes of the Alzheimer’s are not fully understood by the scientific community. Despite several clinical trials, no treatment has been proved to be---even partially---effective in treating the disease. Most scientists agree that probably there is no single cause for Alzheimer’s, but a combination of several genetic and environmental factors.

\fxwarning{Add references}

\subsection{The amyloid hypothesis}

A long standing attempt to explain Alzheimer’s disease is the \emph{amyloid hypothesis}. One of the most distinctive characteristics of brain of people affected by the disease is the formation of aggregates of amyloid-β (Aβ). These aggregates, known as \emph{plaques}, form outside the neural cells and are thought to start a cascade process involving brain inflammation that eventually leads to widespread cell deaths and synapses dysfunction. Despite solid evidence proving the association between Alzheimer's disease and Aβ plaques, the amyloid hypothesis didn't lead to any beneficial result in terms of medical therapy. Various treatments targeting Aβ aggregates have been trialled and some was successful in dissolving plaques, but none had positive and tangible effect on the patients' cognitive functions. The general explanation of these disappointing results is that treatments are administered too late in the disease progression, when plaques have already activated an uncontrollable cascade of damaging processes. Supporting this idea, it has been shown that formation of Aβ plaques can begin decades before the manifestation of symptoms. Despite being often criticised for its failures, the amyloid theory remains one of the main directions in Alzheimer's-related research.

\begin{figure}
  \includegraphics[width=\textwidth]{nia_plaques.jpg}
  \caption{\textbf{Amyloid-β placques in the brain.} }
\end{figure}


\section{Single-particle tracking}
